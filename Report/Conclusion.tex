%!TEX root = Main.tex
\documentclass[Main]{subfiles}

\begin{document}
\chapter{Conclusion} % (fold)
\label{cha:conclusion}

In this course the theory of solving the localization problem for a robot in a known environment was covered.

Six main theory sections were used as the basis for a project that used the theory to solve the localization robot in a custom robot.
A system was build using a chassis, a XV11 LIDAR Sensor, two motors and a Zybo7000 board with accompanying modification boards.
High Level Synthesis was used to produce custom hardware components for motor control in reprogrammable logic.
The Xilinx SDK was used to program software, which was designed in an object-oriented manner.

In the end, a working system was created, though improvements to the motion model would greatly improve the efficiency of the robot.

% chapter conclusion (end)
\end{document}