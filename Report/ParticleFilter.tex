%!TEX root = Main.tex
\documentclass[Main]{subfiles}

\begin{document}
\section{Particle Filter} % (fold)
	\label{sec:particlefilter}
In this is about the Particle Filter. 
Particle Filter can be used for localization by spread particles across the map and then use them as belief. This the last non-parametric method and also the last localization method presented in this course.

The basic idea with the particle filter in a localization application is to choose a number of particles, generate that many particles with random poses, and make them converges towards the robot position.
The basic algorithm is the following:
\begin{enumerate}
\item Generate a particle set $\chi$ of particles with random poses 
\item Calculate the importance weights of the particles 
\item Resample the particle set
\item Move the particles
\item Go to point 2.
\end{enumerate}
As this shows, all the particles have their importance weight calculated at each iteration.
The importance weight is a measure of how well a particle matches the measured position.
This can for an example be calculated using a gaussian distribution:
\begin{equation}
	p(\mathbf{x}|\mathbf{z}) = \frac{1}{\sqrt{2\pi^2 \cdot |\mathbf{C_z}|}} \exp \left( -\frac{1}{2}\cdot (\mathbf{z}-\mathbf{x})^T \mathbf{C_z}^{-1} (\mathbf{z}-\mathbf{x}) \right)
\end{equation}


skriv om particle filter rent generelt
importance sampling
resampling ways
kidnapped robot solutions

	% section introduction (end)

\end{document}