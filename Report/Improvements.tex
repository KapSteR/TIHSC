%!TEX root = Main.tex
\documentclass[Main]{subfiles}

\begin{document}
\section{Improvements} % (fold)
	\label{sec:improvements}

	This section highlights some of the improvements that could be made to the project.
	The main reasons these improvements were not implemented were lack of time, technology or working software licenses.
	
	The motor controller is the first point that should be improved.
	Currently the motor controller works by applying a PWM signal for a certain period of time.
	This works in some circumstances, but factors like an uneven surface makes it very unreliable.
	The motors do contain an odometer which in theory would make it possbile to get a more precise movement, by making a PID controller.
	
	With a more precise movement, the possible turn angle can also be expanded.
	Currently the robot is limited to 90 degree turns, which makes it hard for the robot to navigate back to its path.
	Quantisizing the movement to 10 degrees instead of 90 degrees could make the robot able to make up for small errors in rotation.
	
	In the current motion model a rotation also involves a translation.
	This complicates the motion model, and adds a lot of uncertainty to a rotation.
	This uncertainty could be greatly reduced by changing the motion model to  a pure rotation around the center of the robot.
	This could be done by making the wheels of the robot move in opposite directions when turning.
	Unfortunately, the license of the tool that was used for generating the hardware for the motor controller has expired, making this change impossible.
	
	Another improvement would be to add a wireless connection between the robot and a PC.
	The wireless connection would provide two advantages.
	The first advantage would be monitoring, which would greatly speed up the debugging process.
	Currently the state of the robot is not easily accessible once the system is running.
	This makes debugging very hard, as it requires a physical connection to the device through a cable.
	
	The other advantage would be the ability to control the parameters of the robot. 
	This could involve changing the heuristic or penalty map, changing the position of the goal, or changing the number of particels.
	Currently, implementing any change is a cumbersome process, as it involves re-flashing the device. 
	This takes 5-10 minuts, making the debugging cycles very long.
	
	A wireless connection could be made by buying an additional PMOD for the ZYBO board, which adds support for a wireless connection.
	
	A final improvement would be to get the robot to move more continuously. 
	This would require the update rate of the particle filter to be improved, since it is the current bottleneck of the system.
	Especially the resampling can take a long time, because the current method of resampling takes non deterministic time, with a worst case of $O(n^2)$ time.
	
	If an approach like the tournament approach was used, the resampling would take $O(n)$ time.
	% section introduction (end)
\end{document}