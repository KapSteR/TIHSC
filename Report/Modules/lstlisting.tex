%--------------------------------------------------------------------------%
%------------------------ Defining of colors ------------------------------%
%--------------------------------------------------------------------------%
\usepackage{xcolor}   % Color package

\definecolor{ase_blue}  {RGB}{10, 55,  136}
\definecolor{darkviolet}{rgb}{0.58, 0.0, 0.83}
\definecolor{darkgreen} {rgb}{0.0,  0.2, 0.13}
\definecolor{darkblue}  {rgb}{0.0,  0.0, 0.55}
\definecolor{darkgray}  {rgb}{0.66, 0.66,0.66}
\definecolor{webblue} {rgb}{0.0,  0.05,0.45}
\definecolor{MyDarkBlue}{rgb}{0,  0.08,0.45}
\definecolor{webred}  {rgb}{0.75, 0,   0}


%---------------------------------------------------------------------------%
%------------------------------ LstListing ---------------------------------%
%---------------------------------------------------------------------------%

% Følgende er til koder.
%----------------------------------------------------------
%\begin{lstlisting}[caption=Overskrift på boks, style=Code-C++, label=lst:referenceLabel]
%public void hello(){}
%\end{lstlisting}
%----------------------------------------------------------

% Exstra space
\usepackage{xspace}
% Name on the box followed by \xspace will give the name, a space, and the number.
% Edit title of \codeTitle, if another title is wanted.
% \newcommand{\codeTitle}{Code snippet\xspace} % HERE

% Packages for lstlisting
\usepackage{listings}
\usepackage{color}
\usepackage{textcomp}
\definecolor{lbcolor}{rgb}{0.9,0.9,0.9}

\renewcommand{\lstlistingname}{\codeTitle} % Set the new title of the box

\lstdefinestyle{Code}
{
% aboveskip   = {1.5\baselineskip},
  backgroundcolor = \color{lbcolor},
  basicstyle    = \ttfamily\scriptsize,
  breakatwhitespace= false,
  breaklines    = true,
    columns     = fixed,
  commentstyle  = \color{darkgreen},
  emphstyle   = \color{red}\bfseries,
    extendedchars = true,
  frame       = lines,
  framexrightmargin = 0pt, %6pt
  identifierstyle = \ttfamily,
  keywordstyle  = \color{darkviolet}\bfseries,
  lineskip    = 1pt,
  literate    = {~}{$\sim$}1 {æ}{\ae}1 {ø}{\oe}1 {å}{\aa}1 {Æ}{\AE}1 {Ø}{\OE}1 
{Å}{\AE}1,
  morecomment   = [s][\color{lightblue}]{/**}{*/},
  numbers     = left, % Want numbers? If not, outcomment this line
  numbersep   = 6pt,
  numberstyle   = \footnotesize,
  prebreak    = \raisebox{0ex}[0ex][0ex]{\ensuremath{\hookleftarrow}},
  showstringspaces= false,
  stepnumber    = 2,
  stringstyle   = \color{darkblue},
  tabsize     = 2,
% upquote     = true,
}

%Width needed for the frame - this must be 6 pt
\usepackage{calc}


% Styles for different code languages.
\usepackage{caption}
\DeclareCaptionFont{white}{\color{white}}
\DeclareCaptionFormat{listing}%
{\colorbox[cmyk]{0.43, 0.35, 0.35,0.35}{\parbox{\textwidth - \marginparsep}{\hspace{5pt}#1#2#3}}}

\captionsetup[lstlisting]
{
  format      = listing,
  labelfont   = white,
  textfont    = white, 
  singlelinecheck = false, 
  width     = \textwidth - \marginparsep,
  margin      = 0pt, 
  font      = {bf,footnotesize}
}

\lstdefinestyle{Code-C}   {language = C,      style=Code}
\lstdefinestyle{Code-Java}  {language = Java,     style=Code}
\lstdefinestyle{Code-C++}   {language = [Visual]C++,style=Code}
\lstdefinestyle{Code-VHDL}  {language = VHDL,     style=Code}
\lstdefinestyle{Code-Bash}  {language = Bash,     style=Code}
\lstdefinestyle{Code-Matlab}{language = Matlab,   style=Code}
\lstdefinestyle{Code-Prolog}{language = Prolog,   style=Code}

