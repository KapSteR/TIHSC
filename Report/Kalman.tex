%!TEX root = Main.tex
\documentclass[Main]{subfiles}

\begin{document}
\section{Kalman} % (fold)
	\label{sec:kalman}
This lesson covers the theory about the basic Kalman Filter (KF) and the Extended Kalman Filter (EKF).
\subsection{The Kalman Filter}
The KF is a technique for filtering and predicting a linear system. 
It can be extended to also work on non-linear system, thus the EFK. 
The system in both KF and EKF is presented as a state-space model.
\autoref{eq:ss_state_eq1} and \autoref{eq:ss_state_eq2} show the linear state-space model equations that can be used in the KF.
\begin{equation}
\label{eq:ss_state_eq1}
\mathbf{x}_k = \mathbf{A}_k \mathbf{x}_{k-1} + \mathbf{B}_k \mathbf{u}_k + \mathbf{w}_k
\end{equation}
\begin{equation}
\label{eq:ss_state_eq2}
\mathbf{z}_k = \mathbf{H}_k \mathbf{x}_k + \mathbf{v}_k
\end{equation}
In these equations $k$ is the time index, $\mathbf{x}_k$ and $\mathbf{x}_{k-1}$ are state vectors at the different time instances, $\mathbf{u}_k$ is the control vector, $\mathbf{z}_k$ is the measurement vector, $\mathbf{A}_k$ is the state transition matrix of the system, $\mathbf{B}_k$ is the control matrix, $\mathbf{H}_k$ is the output measurement matrix, $\mathbf{w}_k$ is the process noise vector and $\mathbf{v}_k$ is the measurement noise vector.
The two noise vector are assumed to be Gaussian zero-mean white noise with the covariances $\mathbf{Q}$ and $\mathbf{R}$, where $\mathbf{Q}$ describes the covariance for the process noise and $\mathbf{R}$ for the measurement noise. 
It is also assumed that the noises are uncorrelated, \citep{Simon2006}.

The basic KF operates in a cyclic manner between two states; the time update state and the measurement update state, sometimes also called the prediction state and the correction state. 
These two state can be compared to the ones described in \autoref{sec:localization}; sensing and moving.
Sensing matches the measurement update state, and moving matches the time update state.
This and the following description of the cyclic behavior, is based on \cite{Simon2006}.
The cyclic behavior is shown in \autoref{fig:cyclic_beha_kf}.
\begin{figure}[H]
	\centering
	\includegraphics[width=0.5\linewidth]{./Figures/kf_states.png}
	\caption{The cyclic behavior of the Kalman Filter}
	\label{fig:cyclic_beha_kf}
\end{figure}\noindent
Each of these states consists of several equations. 
The prediction state uses the state-space model to calculate a prediction of the system state, called the a priori state, from the previous corrected system state, called the posterior state, \autoref{eq:kf_state_apriori}.
A priori is marked with a '$^-$', and posterior is marked with a '$^+$'.
\begin{equation}
	\label{eq:kf_state_apriori}
	\mathbf{x}_k^-=\mathbf{A} \mathbf{x}_{k-1}^+ + \mathbf{B} \mathbf{u}_k
\end{equation}
Here it is assumed that the system and control matrices are constant at all time instances.
It also predicts an a priori covariance of the state, that describes how certain the prediction is, \autoref{eq:kf_cov_apriori}.
\begin{equation}
	\label{eq:kf_cov_apriori}
	\mathbf{P}_k^-=\mathbf{A} \mathbf{P}_{k-1}^+ \mathbf{A}^T+\mathbf{Q}
\end{equation}
The correction state then starts by calculating the Kalman Gain. 
The Kalman Gain is a matrix of weights, that show how much belief there is in the prediction and the measurement. \autoref{eq:kf_kalman_gain} is the calculation of it.
\begin{equation}
	\label{eq:kf_kalman_gain}
	\mathbf{K}_k = \mathbf{P}_k^- \mathbf{H}^T (\mathbf{H} \mathbf{P}_k^- \mathbf{H}^T + \mathbf{R})^{-1}
\end{equation}
The Kalman Gain is then used to correct the states as shown in \autoref{eq:kf_state_posterior}, thereby calculation the posterior state. 
\begin{equation}
	\label{eq:kf_state_posterior}
	\mathbf{x}_k^+ = \mathbf{x}_k^- + \mathbf{K}_k (\mathbf{z}_k - \mathbf{H} \mathbf{x}_k^-)
\end{equation}
The a priori covariance is also corrected to the posterior covariance by using the Kalman Gain, \autoref{eq:kf_cov_posterior}.
\begin{equation}
\label{eq:kf_cov_posterior}
\mathbf{P}_k^+ = (\mathbf{I} - \mathbf{K}_k \mathbf{H}) \mathbf{P}_k^-
\end{equation}
$\mathbf{I}$ is an identity matrix.
By looking at these equations, it can be seen that a KF actually describes how the mean and the covariance of the states propagates in time.
This becomes obvious when looking at an 1D example from \citep{Thrun2002}.
\begin{figure}[H]
	\centering
	\includegraphics[width=0.8\linewidth]{./Figures/kf_ex.png}
	\caption{Kalman Filter 1D example, borrow from \citep{Thrun2002}}
	\label{fig:kf_ex}
\end{figure}\noindent

\newpage
In this example a robot is moving in 1D.
\autoref{fig:kf_ex}a shows the a priori state and uncertainty, variance, of the robot.
This mean that the KF is in its prediction state.
The robot measures its position at \autoref{fig:kf_ex}b, and the uncertainty here is described by the measurement noise variance, $\mathbf{R}$.
The correction state calculates the Kalman Gain and finds the posterior state and uncertainty as seen in \autoref{fig:kf_ex}c.
Notice that the uncertainty now has a higher peak and smaller width.
The robot then moves in \autoref{fig:kf_ex}d, thereby adding uncertainty to its position; like it does in \autoref{sec:localization}.
This uncertainty is, as described in \autoref{eq:kf_cov_apriori}, the process noise variance, $\mathbf{Q}$; KF is back in the prediction state.
In \autoref{fig:kf_ex}e, it repeats the process from \autoref{fig:kf_ex}b.
And in \autoref{fig:kf_ex}f, it corrects itself like in \autoref{fig:kf_ex}c.

\subsection{The Extended Kalman Filter}
The EKF is, as mentioned earlier, a KF that takes into account that a system may not be linear, which is the case in most situations.
The state-space model is now present in the form as shown in \autoref{eq:ss_state_eq3} and \autoref{eq:ss_state_eq4}.
\begin{equation}
\label{eq:ss_state_eq3}
\mathbf{x}_k = g(\mathbf{u}_k,\mathbf{x}_{k-1}) + \mathbf{w}_k
\end{equation}
\begin{equation}
\label{eq:ss_state_eq4}
\mathbf{z}_k = h(\mathbf{x}_k) + \mathbf{v}_k
\end{equation}
Where $g$ and $h$ are the non-linear functions, that now describes the transform of the prior state and the control input instead of matrix $\mathbf{A_k}$ and $\mathbf{B_k}$, and the transform of the posterior state instead of $\mathbf{H_k}$ respectively.
The downside of representing a system in the non-linear manner, is that the uncertainties no longer can be presented as Gaussians.
The EKF takes this into account and tries to correct it by doing linearization, which is the key concept of the EFK.
By linearizing the system at every iteration, the uncertainties can again be approximated to be Gaussians.
The EKF linearize by making a First Order Taylor Expansion which calculates a tangent to the system at every iteration.

The EFK algorithm is shown in \autoref{eq:ekf_state_apriori} to \autoref{eq:ekf_cov_posterior}.
\begin{equation}
\label{eq:ekf_state_apriori}
\mathbf{x}_k^- = g(\mathbf{u}_k,\mathbf{x}_{k-1}^+) 
\end{equation}
\begin{equation}
%\label{eq:ekf_cov_apriori}
\mathbf{P}_k^- = \mathbf{G}_k \mathbf{P}_{k-1}^+ \mathbf{G}_k^T+\mathbf{Q}
\end{equation}
\begin{equation}
%\label{eq:ekf_kalman_gain}
\mathbf{K}_k = \mathbf{P}_k^- \mathbf{H}_k^T (\mathbf{H}_k \mathbf{P}_k^- \mathbf{H}_k^T + \mathbf{R})^{-1}
\end{equation}
\begin{equation}
%\label{eq:ekf_state_posterior}
\mathbf{x}_k^+ = \mathbf{x}_k^- + \mathbf{K}_k (\mathbf{z}_k - h(\mathbf{x}_k^-))
\end{equation}
\begin{equation}
\label{eq:ekf_cov_posterior}
\mathbf{P}_k^+ = (\mathbf{I} - \mathbf{K}_k \mathbf{H}_k) \mathbf{P}_k^-
\end{equation}
From these equations it is obvious that they are somewhat like the KF algorithm.
The biggest changes are that the non-linear functions are used instead of the state-space matrices, and that $\mathbf{G}_k$ and $\mathbf{H}_k$ are the Jacobians of the non-linear functions.

	% section introduction (end)

\end{document}