%!TEX root = Main.tex
\documentclass[Main]{subfiles}

\begin{document}
\section{SLAM} % (fold)
	\label{sec:slam}
This lesson covers Simultaneous Localization and Mapping (SLAM). 
SLAM adds another layer of complexity to the problems that have been looked at so far. 
In all previous lessons, the map is assumed to be known. 
In reality that is very seldom the case. 
Therefore the robot not only has to localize itself, it also has to create the map it is localizing itself in.

Noget med hvorfor kortet er vigtigt.

In this lesson a specific form of SLAM is presented called graph SLAM. 
In graph slam, the initial location of the robot, relative movements of the robot and landmark locations are all used as constraints. 
These constraints are combined to find the most likely path the robot has traveled along with the locations of the landmarks, hence SLAM.

Graph slam is a way of reducing SLAM to the solution of a linear system. 
This is done by solving the following equation
\begin{equation}
	\mu = \Omega^{-1} \cdot \xi
\end{equation}
Here $\Omega$ is the graph slam matrix, $\xi$ is the graph slam vector and µ is the estimated robots poses and landmarks. In the approach shown, the correspondences of landmarks are assumed to be known.

	% section introduction (end)

\end{document}