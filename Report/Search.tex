%!TEX root = Main.tex
\documentclass[Main]{subfiles}

\begin{document}
\section{Search} % (fold)
	\label{sec:search}

This lesson covers the planning problem. 
The planning problem is defined as the task of finding the optimal or “minimum cost” path to reach a certain goal position, given a map, a starting location, a goal location and a cost of movement.

The cost of movement can be altered to prioritize specific paths. 
An example could be increasing the cost of moving next to a wall, to prioritize paths away from walls. 

A naïve search algorithm, also known as Breath First Planning, is introduced, where the cost of moving from one cell to another is one. 
The map on which this algorithm will be demonstrated is shown below.
\begin{table}[H]
	\centering
	\begin{tabular}{cccccc}
		R & 1 & 0 & 0 & 0 & 0  \\ 
		0 & 1 & 0 & 0 & 0 & 0  \\ 
		0 & 1 & 0 & 0 & 0 & 0  \\ 
		0 & 1 & 0 & 0 & 0 & 0  \\  
		0 & 0 & 0 & 0 & 0 & G  \\ 
	\end{tabular}
\caption{Grid map for demonstration}
\label{table:grid_map} 
\end{table} \noindent
'0' is an empty cell, '1' is a cell containing a wall, 'R' is the location of the robot, and 'G' is the location of the goal.

The naïve search algorithm iteratively finds the next valid cell with the lowest cost, and expands it. 
The expansion order and the cost are shown below. 
Nodes that have not been visited have a value of '-1'.
\begin{table}[H]
	\begin{subtable}{0.5\linewidth}
		\centering
		\begin{tabular}{cccccc}
			0 & -1 & -1 & -1 & -1 & -1  \\ 
			1 & -1 & 12 & -1 & -1 & -1  \\ 
			2 & -1 &  9 & 13 & -1 & -1  \\ 
			3 & -1 &  7 & 10 & 14 & -1  \\  
			4 &  5 &  6 &  8 & 11 & 15  \\ 
		\end{tabular}
	\caption{Expansion order}
	\label{table:expansion_order} 
	\end{subtable}
	\begin{subtable}{0.5\linewidth}
		\centering
		\begin{tabular}{cccccc}
			0 & -1 & -1 & -1 & -1 & -1  \\ 
			1 & -1 &  9 & -1 & -1 & -1  \\ 
			2 & -1 &  8 &  9 & -1 & -1  \\ 
			3 & -1 &  7 &  8 &  9 & -1  \\  
			4 &  5 &  6 &  7 &  8 &  9  \\ 
		\end{tabular}
	\caption{Cost}
	\label{table:cost_order} 
	\end{subtable}
\caption{Breath First Planning}
\end{table} \noindent
Once the algorithm reaches the goal, after 15 expansions, it stops. 
The path is then found by following the descending g values, the cost value, from the goal position to the start position.
This creates the following path:
\begin{table}[H]
	\centering
	\begin{tabular}{cccccc}
		$\downarrow$ & 1 & 0 & 0 & 0 & 0  \\ 
		$\downarrow$ & 1 & 0 & 0 & 0 & 0  \\ 
		$\downarrow$ & 1 & 0 & 0 & 0 & 0  \\ 
		$\downarrow$ & 1 & 0 & 0 & 0 & 0  \\  
		$\rightarrow$ & $\rightarrow$ & $\rightarrow$ & $\rightarrow$ & $\rightarrow$ & G  \\ 
	\end{tabular}
\caption{Breath First Planning Path}
\label{table:path_map} 
\end{table} \noindent
This algorithm works, but in a large grid this algorithm becomes very ineffective.

Next A-star is introduced. 
A-star works in much the same way as the naïve algorithm, but the cost function is a little more nuanced.
In the naïve algorithm the cost was only based on movement. 
In A-star the cost is the sum of the cost of movement, and a cost introduced by a heuristic function. 
In the lesson a simple heuristic function is proposed, where the distance-to-goal for every point in the map is calculated. 
The heuristic function for the presented map would look like this:
\begin{table}[H]
	\centering
	\begin{tabular}{cccccc}
		9 & 8 & 7 & 6 & 5 & 4  \\ 
		8 & 7 & 6 & 5 & 4 & 3  \\ 
		7 & 6 & 5 & 4 & 3 & 2  \\ 
		6 & 5 & 4 & 3 & 2 & 1  \\  
		5 & 4 & 3 & 2 & 1 & 0  \\ 
	\end{tabular}
\caption{Heuristic map}
\label{table:heuristic_map} 
\end{table} \noindent
This means the cost of cells close to the path is lower, than the cost of cells far away from the goal.
The algorithm then explores the nodes with the shortest distance-to-goal first. 
This greatly reduces the number of nodes that have to be explored, and thus makes the algorithm more efficient. 
The cost and the expansion order of A-star in the map from before is shown below.
\begin{table}[H]
	\begin{subtable}{0.5\linewidth}
		\centering
		\begin{tabular}{cccccc}
			0 & -1 & -1 & -1 & -1 & -1  \\ 
			1 & -1 & -1 & -1 & -1 & -1  \\ 
			2 & -1 & -1 & -1 & -1 & -1  \\ 
			3 & -1 & -1 & -1 & -1 & -1  \\  
			4 &  5 &  6 &  7 &  8 &  9  \\ 
		\end{tabular}
	\caption{Expansion order}
	\label{table:expansion_order_heuristic} 
	\end{subtable}
	\begin{subtable}{0.5\linewidth}
		\centering
		\begin{tabular}{cccccc}
			9 & -1 & -1 & -1 & -1 & -1  \\ 
			9 & -1 & -1 & -1 & -1 & -1  \\ 
			9 & -1 & -1 & -1 & -1 & -1  \\ 
			9 & -1 & 10 & 10 & 10 & -1  \\  
			9 &  9 &  9 &  9 &  9 &  9  \\ 
		\end{tabular}
	\caption{Cost}
	\label{table:cost_order_heuristic} 
	\end{subtable}
\caption{A-star}
\end{table} \noindent
This creates a path, which is exactly the same as the Breath First Planning, \autoref{table:path_map}.

While the path of the naïve algorithm and A-Star is the same, the path finding algorithm of A-Star is much more efficient, as it only has to do 10 expansions, compared to the naïve approaches 15.

The last planning method is based on dynamic programming. 
Here the optimal path for any location on the map is pre-calculated. 
This might be very computationally heavy up front, but it only has to be done once. 
Dynamic programming would produce a map like the one below.
\begin{table}[H]
	\centering
	\begin{tabular}{cccccc}
		$\downarrow$ & 1 & $\downarrow$ & $\downarrow$ & $\downarrow$ & $\downarrow$  \\ 
		$\downarrow$ & 1 & $\downarrow$ & $\downarrow$ & $\downarrow$ & $\downarrow$  \\ 
		$\downarrow$ & 1 & $\downarrow$ & $\downarrow$ & $\downarrow$ & $\downarrow$  \\ 
		$\downarrow$ & 1 & $\downarrow$ & $\downarrow$ & $\downarrow$ & $\downarrow$  \\  
		$\rightarrow$ & $\rightarrow$ & $\rightarrow$ & $\rightarrow$ & $\rightarrow$ & G  \\ 
	\end{tabular}
\caption{Map for dynamic programming}
\label{table:dynamic_map} 
\end{table} \noindent
	% section introduction (end)

\end{document}