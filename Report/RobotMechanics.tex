%!TEX root = Main.tex
\documentclass[Main]{subfiles}

\begin{document}

\section{Robot Mechanics} % (fold)
	\label{sec:robot_mechanics}

	\subsection{Chassis} % (fold)
	\label{sub:chassis}
		The chassis of the robot is a Baron-4WD mobile platform bought from the website dfrobots.com. 
		It consist of a lower frame where up to four DC motors can be attached, and a top plate with holes for mounting of other devices. 
		\autoref{fig:baron_platform} shows a picture of the chassis where it is assembled with the default package items from dfrobots \cite{DFRobot2015}.
		\begin{figure}[H]
			\centering
			\begin{subfigure}[b]{0.55\linewidth}
				\includegraphics[width=1\linewidth]{./Figures/baron_platform.jpg}
				\caption{The Baron-4WD platform}
				\label{fig:baron_platform}
			\end{subfigure}	
			\begin{subfigure}[b]{0.4\linewidth}
				\includegraphics[width=1\linewidth]{./Figures/final_robot.png}
				\caption{The final robot configuration}
				\label{fig:final_robot}
			\end{subfigure}
		\caption{}
		\label{fig:robots}
		\end{figure}
		In this project it was decide to only use two motors at the rear end of the chassis to make a simpler motion model for the robot; more about the motion model in \autoref{sub:motion_model}. 
		Instead of front wheels, a ball caster was attach to insure the turning ability. 
		By only using motors in the rear end, the lower frame had a lot of room where the power supplies could be place and thereby leaving the top plate free for the processing platform, a Zynq development board, \autoref{sub:platform}, and mounting of the sensors. 
		The sensor used in this project is, as mentioned earlier, a LIDAR which required a specific mounting pattern, that filled out most of the top plate's area, thereby only leaving a small area for the ZYBO. 
		More about the LIDAR in \autoref{sub:sensor}.
		The LIDAR was therefore mounted on a acrylic plate which then was mounted on top of the top plate. 
		The final robot configuration is shown in \autoref{fig:final_robot}.
		
		% subsection chassis (end)

	\subsection{Motor characteristics} % (fold
	\label{sub:motor_characteristics}

		In order to have an estimate of wheel speed related to input voltage for use when designing a motion model (see \autoref{sub:motion_model}), a small experiment was carried out.
		In this, a motor was powered in the same way i would be in the final robot (see \autoref{sub:motor_control}) and the steady-state rotation speed was recorded for different duty-cycles.
		The results of this is shown in \autoref{fig:motor_speed_test} below.

		\begin{figure}[H]
			\centering
			\includegraphics[width=\linewidth]{MotorSpeedTest}
			\caption{Motor rotational velocity [RPM] vs. PWM duty cycle [\%]}
			\label{fig:motor_speed_test}
		\end{figure}

		\autoref{fig:motor_speed_test} shows that the speed of the motor, overall, changes linearly with input voltage.
		It is, however, apparent that at very low voltage the torque is not sufficient to get the motor to rotate.

		The two major takeaways from this experiment and motor theory in general is:
		\begin{enumerate}
			\item
			It is most optimal to operate in the mid to high range of input voltage, as this is the most linear region of operation.

			\item
			The motor input voltage may preferably be controlled by a PID controller to set a certain speed, instead of an open loop approach where transient acceleration due to inertia is more pronounced.
		\end{enumerate}

		% motor characteristics (end)

	\subsection{Motion model} % (fold)
	\label{sub:motion_model}

		The motion model used in this project is based on a simple rotation and the move model.
		The rotation is however not just a change in orientation since the robot is not rotating around it's center of mass, but around one of it's wheels.
		This means that during rotation the center of mass is also being moved.
		The difference between these two rotations is shown in \autoref{fig:motion_models}.
		\begin{figure}[h]
			\centering
			\begin{subfigure}[b]{0.45\linewidth}
				\includegraphics[scale=0.8]{./Figures/motion_model1.png}
				\caption{Around center of mass}
				\label{fig:motion_model1}
			\end{subfigure}	
			\begin{subfigure}[b]{0.45\linewidth}
				\includegraphics[scale=0.8]{./Figures/motion_model2.png}
				\caption{Around wheel}
				\label{fig:motion_model2}
			\end{subfigure}
		\caption{Rotation models}
		\label{fig:motion_models}
		\end{figure}
		Where the thin lined figures are the robot before rotation, and the thick figures are after.

		To describe how the center of mass is being move when turning around the wheel, the following equation from \cite{Wikipedia2015} is used, which is just a 2D rotation matrix.
		\begin{equation}
		\begin{bmatrix} x' \\ y' \end{bmatrix} = \begin{bmatrix} \cos{\theta} & -\sin{\theta} \\ \sin{\theta} & \cos{\theta} \end{bmatrix} \begin{bmatrix} x \\ y \end{bmatrix}
		\end{equation}
		In this equation $x$ and $y$ are the coordinates before the transformation, $x'$ and $y'$ are after transformation, and $\theta$ is the angle of how much the orientation is being changed.

		The movement is described simple by the change in the coordinates, \autoref{eq:translation_matrix}.
		\begin{equation}
		\label{eq:translation_matrix}
		\begin{bmatrix} x' \\ y' \end{bmatrix} = \begin{bmatrix}  d & 0 & x \\ 0 & d & y \end{bmatrix} \begin{bmatrix} \cos{\phi} \\ \sin{\phi} \\ 1 \end{bmatrix} 		
		\end{equation}
		Where $x$, $y$, $x'$ and $y'$ is the same as previous, $\phi$ is the orientation and $d$ is the distance of the movement.
		
		To set a certain rotation ($\theta$) or distance ($d$) one or two motors are set to run at a certain speed, and for a time linearly proportional to the change.
		This motion model assumes constant velocity, which means that the robot accelerates to its final speed instantly.
		This is of course impossible, but the assumption does break the model completely as the speed we are moving at is very low and so is the inertia of the robot.
		
		Preferably the motor speed would be controlled by a PID controller, in order to have maximum torque when accelerating, making the true robot motion more similar to the model.
		This is however not implemented in this project because of time constraints.
		In stead the motion suffers from noise caused by the transient nature of the open loop control.

		It was chosen to use this discrete model, where rotation and translation (movement) are done separately, above a continues model, where rotation is being done while translating.
		The reason for this was to keep model motion as simple as possible to begin with, and then extend it when a functional robot was made.

		For simplicity, it was also decided that the change in orientation would by constrained to intervals of $90\degree$.
		This meant that the robot in theory should only be able to drive horizontally and vertically in the grid map.

		% subsection motion_model (end)

	% subsection robot_mechanics (end)

\end{document}